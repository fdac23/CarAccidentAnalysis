%%%%%%%%%%%%%%%%%%%%%%%%%%%%%%%%%%%%%%%%%%%%%%%%%%%%%%%%%%%%%%%%%%%%%%%%%%%%%%%%
%2345678901234567890123456789012345678901234567890123456789012345678901234567890
%        1         2         3         4         5         6         7         8

\documentclass[letterpaper, 10 pt, conference]{ieeeconf}  % Comment this line out
                                                          % if you need a4paper
%\documentclass[a4paper, 10pt, conference]{ieeeconf}      % Use this line for a4
                                                          % paper

\IEEEoverridecommandlockouts                              % This command is only
                                                          % needed if you want to
                                                          % use the \thanks command
\overrideIEEEmargins
% See the \addtolength command later in the file to balance the column lengths
% on the last page of the document

\usepackage[utf8]{inputenc}
\usepackage[T1]{fontenc}

% The following packages can be found on http:\\www.ctan.org
%\usepackage{graphics} % for pdf, bitmapped graphics files
%\usepackage{epsfig} % for postscript graphics files
%\usepackage{mathptmx} % assumes new font selection scheme installed
%\usepackage{mathptmx} % assumes new font selection scheme installed
%\usepackage{amsmath} % assumes amsmath package installed
%\usepackage{amssymb}  % assumes amsmath package installed

\title{\LARGE \bf
Car Accident Analysis Project Proposal
}

\author{Noah Shoap and Jonathan Graham
}


\begin{document}



\maketitle
\thispagestyle{empty}
\pagestyle{empty}


%%%%%%%%%%%%%%%%%%%%%%%%%%%%%%%%%%%%%%%%%%%%%%%%%%%%%%%%%%%%%%%%%%%%%%%%%%%%%%%%
\begin{abstract}

A project proposal for the analysis of car accident data sets, with the goal of finding trends we wouldn't expect.

\end{abstract}


%%%%%%%%%%%%%%%%%%%%%%%%%%%%%%%%%%%%%%%%%%%%%%%%%%%%%%%%%%%%%%%%%%%%%%%%%%%%%%%%
\section{Project Objective}
The objective of this project, Car Accident Analysis, is to apply what we've learned as computer scientists and conduct a thorough analysis of a data set containing car accident information.  By doing this, we aim to gain insight into what features of the data set correlate to more frequent and serious accidents, specifically, the aim being to find correlations we may not expect.

\section{Motivation}

Car accidents and the fatalities that often result from them are a serious public safety concern.  While some conditions are obviously bad for driving, such as intense rainfall, perhaps there are some other dangerous correlations that are not as obvious.  If we could find those, bringing awareness to those correlations could make drivers more aware and increase public safety.

\section{Data}

The data we're going to be using is a data set from Kaggle.  It is a rather large data set (the file is over 3GB) and it contains information pertaining to weather conditions, traffic conditions, time of day, city, state, and more.  


The data set is rather large, so we will have a team discussion about if we want to filter it down to a subset of the data, and if so what criteria we will use to form that subset.

\section{Team Responsibility}

We only have two team members, so currently the expectation is that both Jon and Noah will often be working together and on the same tasks.  Currently, the expected tasks are:

\begin{itemize}

\item Data Collection
\item Filtering data set down to a sub set (if we decide to)
\item Analyzing Data
\item Data Visualization
\item Write-up of results

\end{itemize}

\newpage

\section{Milestones}

\begin{itemize}
    \item (Oct. 1) -- Decide on any possible data filtering criteria
    \item (Oct. 8) -- Data used for analysis is finalized
    \item (Oct. 22) -- Analysis Complete
    \item (Nov. 5) -- Visualization of Data / Analysis Complete
    \item (Nov. 12) -- Summary of results written
\end{itemize}

\section{Expected Outcome}
We expect to see some obvious correlations, such as heavy rainfall correlating to more frequent accidents.  But we also expect that we will find some correlations that we did not expect going into the analysis.  The hope being that there is something we can learn from this analysis to help people become safer drivers.

\addtolength{\textheight}{-12cm}   % This command serves to balance the column lengths
                                  % on the last page of the document manually. It shortens
                                  % the textheight of the last page by a suitable amount.
                                  % This command does not take effect until the next page
                                  % so it should come on the page before the last. Make
                                  % sure that you do not shorten the textheight too much.

%%%%%%%%%%%%%%%%%%%%%%%%%%%%%%%%%%%%%%%%%%%%%%%%%%%%%%%%%%%%%%%%%%%%%%%%%%%%%%%%


\end{document}

